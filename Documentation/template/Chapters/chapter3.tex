\chapter{Requirements}
\label{chapter3}

In this section, requirements of the project are outlined. This is to easily define what problems and goals are to be solved by the project. Factors that are considered for the development of the project are mentioned in the feasibility assessment.

\section{Client Requirements}

<<<<<<< HEAD
For this project the client has requested that a technology demonstration is made that meets their specified requirements. The requirements that were given by the client were to appeal to the target audiences (who will be discussed in the next subsection), and to fully utilise the functionality of the HTC Vive in order to properly demonstrate its capabilities. The client needed this piece of software as they wished to have a demonstration for HTC Vive using software made by University of Leeds students, which they could show to applying students.
=======
The client has asked us to produce this technology demonstration in order to have a piece of software for the HTC Vive to demonstrate to students and has been developed by a University of Leeds student. The requirements that were given by the client were to appeal to the target audiences (who will be discussed in the next subsection), and to fully utilise the functionality of the HTC Vive in order to properly demonstrate its capabilities.
>>>>>>> 7524012b1f509cff1c4e374fd07106dbad642479

\subsection{Target Audience}
The demo would be targeted towards potential students looking to apply to the University of Leeds, as it will be shown to the students on open days in order to gain interest from them.
It would also have to appeal to people and companies in the gaming industry. As the more interest the School of Computing can gain from them, the more potential projects they may have for the school.
\newline
\par
With these target audiences in mind the demo should be technical to impress the gaming industry, while balancing it with being interesting for the potential students. This involves the implementation of features that are not trivial and not already built in the game engine.

\section{Feasibility Assessment}
\subsection{Feasibility}
To support the development of the project, a reserved space where the HTC Vive can be permanently set up for the duration of the project is required. This reserved space would ideally be a room that meets the space requirements stated above for room-scale experiences, so that all the capabilities of the Virtual Reality hardware can be used. A computer which meets the hardware requirements is also needed in order to run the HTC Vive software, This computer must be running Windows since HTC Vive currently only supports this operating system \cite{vivenolinux}. Also, a copy of Unreal Engine 4 game engine must be installed which is free to be downloaded. Unreal Engine 4 is chosen over Unity since the members of the group are more familiar with developing in C++ which Unreal Engine supports rather than C\# which Unity supports, although both of these game engines provide native support for virtual reality developments.
\newline
\par
A possible solution for meeting the hardware requirements is to use a personal machine. A laptop was available with the specifications below which just meets the requirements for the graphical power. Development can be done on our own personal Windows machines and can be tested with the Vive using the laptop in the reserved room. The Vive is not required for conducting simple tests, but it is needed for identifying issues such as scaling and user input with motion controllers.

\subsection{Technical Specifications}

\subsubsection{Hardware Requirements}
\begin{itemize}
	\item Graphics card: NVIDIA GeForce GTX 970 /Radeon R9 280 equivalent or greater
	\item Processor: Intel Core i5-4590 equivalent or greater
	\item RAM: 4GB+
	\item Video Ports: HDMI 1.4, DisplayPort 1.2 or newer 
	\item 1x USB 2.0 port
	\item Room-Scale Space: 2 meters by 1.5 meters
\end{itemize}

These requirements can be found on the official HTC Vive page on the Steam Store \cite{vivehardware}