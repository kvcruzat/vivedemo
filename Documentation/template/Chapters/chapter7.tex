\chapter{Testing and Evaluation}
\label{chapter7}

This chapter outlines how testing was conducted during the development. The overall project and solution produced was then evaluated to see if the goals created was met.

\section{Regular Testing}
Throughout the development, the demo was regularly tested. Aside from the usual testing from the members of the group during development, the weekly meetings with the supervisor provided a medium where new features implemented can be tested. Since the supervisor did not have knowledge of a feature was implemented, there was no bias in making the feature work like it was implemented. This helped with spotting bugs as the player may behave in ways that are not expected. 
\newline
\par
Also, the demo was tested with various other people such as colleagues of the supervisor. People with a range of skill level and familiarity with the hardware tested the solution produced. This provided a good insight to the accessibility of the demo with people that are not familiar with virtual reality. Some people found it difficult to get used to the unfamiliar controller and interaction whereas most people quickly learned especially with the natural grip gesture of holding a disc using the trigger button. The most important aspect that was deduced from the testing was the fact that every tester was quickly immersed in the world and gameplay of the demo. This meant that the world created was interesting and the gameplay aspect of it added to their enjoyability as goal was set for them to complete.
\newline
\par
In summary with the regular testing conducted, it demonstrated that demo created is interesting and immersive to people. The virtual world the player is in helps them experience the capabilities of virtual reality and the added game goal gave the player to opportunity to navigate the world and complete the level. The pitfall of the demo was that the player had to be briefed before hand on the controls and game mechanic especially if they are not familiar with virtual reality. The demo was not easy enough for someone with no prior knowledge to just play. Since the demo was made for open days, this will not be a problem since someone will be there to provide the information, but a way of showing controls or game instructions in game would be an improvement that could be made.

\section{Testing Against Requirements}
	The original requirements that the client had specified were:

\begin{itemize}
	\item Create a technology demonstration for the HTC Vive
	\item Make the technology demonstration appealing to students applying to the University of Leeds.
	\item The technology demonstration has to be appealing to members of graphics industry.
\end{itemize}

	At least two out of three of these requirements have been met. A technology demonstration has been created for the HTC Vive, this has been shown and proven to the client. The technology demonstration has also been shown to be appealing to a member of the graphics industry, this was tested by showing the technology demonstration to a member from the graphics industry. The only requirement that has not been tested is if the demonstration is appealing to applying students, this has not been tested as there has been no opportunity to show the demonstration to an applying student.

\subsection{Student Testing Plan}
	To test if the demonstration is appealing to applying students, the demonstration should be shown to ten students currently studying Computer Science at the University of Leeds, these students should ideally have varying amount of proficiency in using Virtual Reality hardware and also be from different years of the course. During the demonstration to the students several things will be noted, including how quickly they pick up on the game controls and how quickly they figure out the objective of the game. This will be used to evaluate how intuitive the demonstration is. This is important to know as if it is not intuitive enough, then a small explanation will need to be added to the game, either as a small text introduction, or a person telling the students how to play. After the student has completed the demonstration they will be asked to take a small survey containing questions about what they thought about the demonstration, and if they were shown this on an open day would they have a higher chance of picking the University. If most of the surveys contain positive responses, then the demonstration has passed that requirement.

\section{Client Evaluation}
	The client has been shown the technology demonstration and is pleased with how the demonstration has progressed. The client was shown a final version of the product and had no major criticism about the demonstration. It was demonstrated that all features and tasks that was set out to be completed was met with a few extra features such as the campus sky boxes.

\section{Project Evaluation}
\subsection{Schedule}
	For this project the schedule was fairly accurate in terms of how long everything would take to do. There were three separate occasions when the schedule was not held to, but the project was back on track the week after. These three occasions were:
	\begin{itemize}
		\item The Virtual Reality features were not done by the week that they were supposed to be done. This was due to the fact that there was no room for the Vive to be set up in, and therefore the features could not be checked as the Vive could not be used.
		\item The Tower of Hanoi logic was also not finished in the time designated by the schedule. This was due to a bug being in the code that was hard to debug.
		\item Adding water to the graph had to be delayed compared to the schedule due to the fact that the original idea we had for the water flow (placing a flat sheet of water under the graph) would not work, and a more complex solution needed to be made. The original solution did not work as each river needs to have separate flow running through it and this is not possible with all the rivers being one sheet of water. This task was pushed back further in the schedule.
	\end{itemize}

	Most of the milestones for the project were met, only the first milestone was not met. This was due to the fact that a room for the Vive was not available until the week before the milestone, so the Virtual Reality features were not implemented properly.

\subsection{Additional Features}
	No additional features were implemented in the project, this was due to the fact that the baseline features were sufficiently challenging to implement, and therefore no time was available to add the additional features.

\subsection{Difficulties with the Project}
	One of the biggest difficulties with this project was the fact that only one Vive was available, and there was only one computer to develop on the Vive. This was solved by having each collaborator work on a separate part of the project, usually one would be working on the Vive using Unreal Engine, and the other would be work on the Out-of-Engine features, i.e. Randomly Generating Graphs. If the only feature that needed to be developed was in Unreal, the other member would help when a problem arises or would work on the other aspects of the project such as documentation.
	
\subsection{Meetings}
	Throughout the project we had weekly meetings with the client to discuss the progress of the project, and to show the current state of the project. These were immensely helpful to help keep the project in time with schedule, and they also helped with making sure each feature was up to the standard of the client.
	During the project the collaborators also had regular meetings in order to test the features if the demonstration, to make sure that everything worked as it should. During these meetings the collaborators would also discuss the plans for the week, and the tasks would be distributed out, and a short discussion would be had about the best way to tackle the tasks for that week.