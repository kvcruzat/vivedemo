\begin{appendices}
\chapter{External Material}

\section{Instruction Manual}

	This section will go over how to install the project, how to compile, how to run it and how to create a new level.

\subsection{Installing the Project}

\begin{enumerate}
	\item Download the project from GitHub. (https://github.com/kvcruzat/vivedemo)
	\item Download and install Unreal Engine. (https://www.unrealengine.com/)
	\item Launch Unreal Engine.
	\item Click the Browse button in Unreal and navigate to ~/RiversofHanoi and open RiversofHanoi.uproject
\end{enumerate}

\subsection{Compiling the Project}

\begin{enumerate}
	\item Open up Level 1 by going to File->Open Level   and then selecting Level1 under the levels directory.
	\item Make sure the terrain is in the level. (unzip terrain.m files in ~/RiversohHanoi/Content/Levels/data/LevelX/terrain.zip)
	\begin{enumerate}
	    \item If the terrain is not on the level then you must drag it in.
		\item This is done under the modes tab, that will be on the left side of the window in the default Unreal layout.
		\item Select the place tab if not selected, and then search for "Terrain Actor", this should produce one result.
		\item Drag this result into the game screen and the terrain should place itself down.
	\end{enumerate}
	\item Repeat step 2 for every level.
	\item To build the project go to File->Package Project->Windows->Windows (64-bit)
	\item Choose where the project should build to.
\end{enumerate}

\subsection{Making a New Level}

\begin{enumerate}
	\item Navigate to ~/Randomly Generate Graphs/terrainGen
	\item Open a command line window here
	\item In the command line type "make", if no make utility is installed then install one, for example CMake (https://cmake.org/)
	\item After the project is compiled then type "graphGen.exe".
	\item This will produce the required files in ~/RiversofHanoi/Content/models
	\item To make a new level, navigate to ~/RiversofHanoi/Content/Levels
	\item Copy Level1.umap to a new file and rename it to next sequential level.
	\item Navigate to ./data and then copy and rename the folder "Level1" to the same name.
	\item Navigate to ~/RiversofHanoi/Content/models and copy the following files
	\begin{itemize}
		\item connections.txt
		\item nodeConnections.txt
		\item nodes.txt
		\item rivers.txt
		\item rodIndex.txt
		\item rods.txt
		\item terrain.m
	\end{itemize}
	\item Move these files to ~/RiversofHanoi/Content/Levels/data/levelx, where x is the number that it was renamed to before.
	\item Open the level in Unreal Engine by going to File->Open Level and selecting the new level.
	\item Check so that everything looks right and that the terrain model is in the game.
\end{enumerate}

\subsection{Running the Game}

\begin{enumerate}
	\item To run the game navigate to the folder that it was built to.
	\item Then there will be a file called RiversofHnaoi.exe, run this file.
\end{enumerate}

\subsection{Playing the Game}

\subsubsection{Controls}

\begin{itemize}
	\item Move head to look around the world.
	\item Teleport: Press and hold on to the trackpad then use the motion controller to point to target destination.
	\item Grab Discs: Press and hold trigger when hand is near a disc to grab.
	\item Let Go Discs: Release the trigger to let go of held disc.
\end{itemize}

\subsubsection{Game Objective}

\begin{itemize}
	\item River starts at a source on a corner of the terrain at the highest point of the terrain.
	\item River flow splits in to half at each split.
	\item Flowers around nodes indicate status of flowers according adjacent river flow.
	\begin{itemize}
		\item White: Too little flow
		\item Blue: Too much flow
		\item Orange: Correct flow
	\end{itemize}
	\item Three different disc sizes when placed on a rod reduces flow of river:
	\begin{itemize}
		\item Small: Reduction of \(\frac{1}{6}\) of current flow
		\item Medium: Reduction of \(\frac{1}{3}\) of current flow
		\item Large: Reduction of \(\frac{1}{2}\) of current flow
	\end{itemize}
	\item Place discs on rods to change flow of rivers to make all flowers turn orange.
	\item Only discs which are larger can be placed on top of smaller discs (Reverse Towers of Hanoi).
	\item Hint: Start with rods closer to the source then work your way down the graph.
\end{itemize}
\chapter{Ethical Issues Addressed}
There were no ethical issues to address.
\end{appendices}